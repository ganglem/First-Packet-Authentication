\section{Conclusion}
\label{sec:conclusion}

Cybersecurity is a highly complex and layered topic.  Each of the discussed techniques is a way to secure and protect data, and every method has its advantages and disadvantages.  \par

Even though the technologies mentioned earlier have not evolved to become the conventional approach to security, their foundation on explicit trust rather than implicit trust is an approach that should be established more often and become the norm in today's highly digitalized world.  \par
It is essential to note that there is no one method to secure a system; every system has its vulnerabilities. Each component that plays a role in security needs to be as strong and unreachable as possible, for the security is dependent on its weakest link.  It always depends on the security architecture, the security experts, and the people who use cybersecurity methods to state which approach shall be used for what purpose to ensure that no malice finds its way into a system.\\\par 

Port Knocking has shown to be a good foundation for the later and improved-on Single Packet Authorization (or better: 'Authentication'). However, Single Packet Authorization fell short as the current most popular implementation depends on a single tool that, as or right now, is not part of the standard Linux kernel.\par

BlackRidge Technology's First Packet Authentication, with the help of their Transport Access Control, improves the already existing TCP protocol and can even be expanded to other technologies. It shows that an already well-established system has the potential to improve and make this zero trust approach the new standard for TCP.\par

Although TAC provides security within the system, unlike Port Knocking and Single Packet Authorization, none of the approaches provides protection to the user initiating a connection to the secured system. Zero trust needs to take place at all levels. Nevertheless, these mechanisms offer a step in the right direction.\\\par

In conclusion, a layered approach resulting with defence in depth and security through obscurity, in combination with the zero trust security design, proposes a solid strategy in cybersecurity.\\\par
